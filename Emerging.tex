\section{Emerging Foveated Rendering Techniques}
The most recent paper on foveated rendering is by \cite{Patney} the paper detailed how they applied a gaze tracking device to a virtual reality headset, for the purpose of developing a new foveated rendering system. The authors discuss in the conclusion of this paper that the reason they set out to create a new rendering technique was because they found that their implementations of both the MULTIRES system by \cite{Guenter:2012:FG:2366145.2366183} and the Coarse pixel shading system by  \cite{Vaidyanathan:2014:CPS:2980009.2980011} introduced artifacts and then realised that they did not have a way of reducing the artifacts. The paper gives an overview on human peripheral vision talking about the various tasks that are at play in our peripheral vision and how these contribute to the effectiveness of foveated renderers. They discuss how recent foveated renderers build on the cortical magnification theory decreasing various visual factors such as resolution, shading rate and screen-space ambient occlusion with increasing eccentricity, and say how this alone fails to describe all aspects of peripheral vision.

They created a "emulated foveated renderer" which was a perceptual sandbox where they performed foveation as a post process, this was so that they could figure out what their perceptual visual target was, which would then be used in their design decision for the final foveated renderer. The way this was evaluated was through a user study. Through the sandbox they were able to emulate three different foveation techniques the first was 'Aliased Foveation', they then emulated 'Stable Foveation', and the final foveation strategy they emulated was the foveation technique they had developed.

They compared each strategy in a user study. The procedure of the study was to show the users various scenes rendered with and without foveation and to pick which looked better, in each trial they sequentially presented a foveated and non-foveated version of the scene in a random order. The results of the user study showed that the threshold for their strategy was significantly higher than the other two being 3x better than aliased foveation and 2x better than temporally stable foveation, therefore they were able to confirm that temporally stable and contrast preserving peripheral images are superior to temporally unstable and non-contrast preserving strategy's. 

Following the results they then produced a foveated renderer for VR that achieves variable-rate sampling without temporal aliasing, preserves contrast and improves performance from reduced sampling rates in the periphery. Following this they performed a second user study to see if the rendering system they had created did in fact achieve superior image quality. They compare the rendering system against the MULTIRES foveated rendering system. they found that their rendering system was capable of generating images of a higher quality than the MULTIRES system, which verified their hypothesis. They also found that their system achieved a 50\% reduction in pixel writes than the MULTIRES system and was 50\% more effective at lowering the shading cost 

Despite the results of the user study being exceptionally positive, there are still limitations in fully realising the potential of foveated rendering in VR. For example the blink detection in the gaze tracker was poor, which then introduced artefacts when blinking due to brief loss of gaze tracking. Although they developed a perceptual target they don’t claim it to be an ideal target and state that future improvements to the perceptual target could provide additional insights into a reduction in shading work.

In the conclusion the authors talk about how they implemented both the MULTIRES system by \cite{Guenter:2012:FG:2366145.2366183} and the coarse pixel shading system by \cite{Vaidyanathan:2014:CPS:2980009.2980011} and found that both introduced objectionable artifacts that were gaze and head motion dependent